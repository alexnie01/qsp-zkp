\documentclass{article}
\usepackage[utf8]{inputenc}
\usepackage{amssymb}
\usepackage{amsmath}

\title{Identrix ZKP}
\author{Alexander Nie}
\date{August 2018}

\begin{document}

\maketitle

\section*{ZKPs in Identrix}

For its zkSNARK, Identrix will use a modified version of the Quadratic Spanning Program (QSP) Problem. Briefly, this problem involves taking a set of polynomials $v_i,w_i, t\big|_{i=1}^m$ over a field $\mathbb{F}$ and finding a linear combination of the $v$ and $w$ polynomials which forms a multiple of the target polynomial $t$. That is, we want to find the elements $a_i,b_i$ of the field $\mathbb{F}$ such that the elements multiplied by the polynomials $v_i, w_i$ give a multiple of $t$. In mathematical notation,

\begin{align}
&\text{Let } m\in \mathbb{Z}: \forall i, 1\leq i \leq m, v_i,w_i,t\in \mathbb{F}[x]\\
&\text{Let } a_i,b_i\in \mathbb{F}, h\in \mathbb{F}[x] \text{ and let } u = \{a_i,b_i\})_{i=1}^m \text{ for some choice of } a_i, b_i\\
&\exists h\in F[x] \text{ and } u : v_a(u)w_b(u)=\left(\sum_{i=1}^m a_i v_i\right)\left(\sum_{i=1}^m b_i w_i\right) = ht\\
&\text{Problem: find one such value of } u
\end{align}

The QSP problem belongs to a group of problems known as \textbf{NP}-complete, which means all other problems in the set \textbf{NP} (most cryptography problems) can be rephrased as a QSP using a \textit{reduction function} $f$ which runs in polynomial time. This is important for Identrix because the question "Can Authenticator prove Bob over 25 years old?" can be expressed as a logic circuit for the SAT problem, which is in \textbf{NP}, so $f$ allows us to transform this into the QSP question "Can Authenticator show me a combination of $a_i,b_i$ to make the polynomials equal?". The specific reduction function $f$ will depend on the context of the statement to be proved (it will vary by dApp).

Two improvements features distinguish the Identrix ZKP scheme from traditional ZKPs. First, rather than performing polynomial multiplication on the $v_i,w_i$, an evaluation point $x_0$ is chosen so the Verifier/Service Provider only needs to check

\[\left(\sum_{i=1}^m a_iv_i(x_0)\right)\left(\sum_{i=1}^m b_iw_i(x_0)\right)=h(x_0)t(x_0)\]

which is a number rather than an algebraic expression. Second, the polynomials are encoded using a homomorphic transformation along an elliptic curve. Essentially, this is a transformation $E$ with the property $E(xy)=E(x)E(y)$ which allows us to encode the test-point $x_0$ and the polynomials since $E(a_iv_i(x_0))=a_iv_i(E(x_0))$. A further level of obfuscation is added by multiplying by a random constant $\alpha$ so that the Service Provider cannot figure out the $a_i,b_i$ supplied by the Prover (instead they receive $\alpha a_i$, $\alpha b_i$). Then the equality must still hold for the proof to be true:

\[\left(\sum_{i=1}^m a_i E(\alpha v_i(x_0))\right)\left(\sum_{i=1}^m b_i E(\alpha w_i(x_0))\right) = E(\alpha h(x_0))E(\alpha t(x_0))\]



\end{document}
